% Options for packages loaded elsewhere
% Options for packages loaded elsewhere
\PassOptionsToPackage{unicode}{hyperref}
\PassOptionsToPackage{hyphens}{url}
\PassOptionsToPackage{dvipsnames,svgnames,x11names}{xcolor}
%
\documentclass[
  letterpaper,
  DIV=11,
  numbers=noendperiod]{scrartcl}
\usepackage{xcolor}
\usepackage{amsmath,amssymb}
\setcounter{secnumdepth}{-\maxdimen} % remove section numbering
\usepackage{iftex}
\ifPDFTeX
  \usepackage[T1]{fontenc}
  \usepackage[utf8]{inputenc}
  \usepackage{textcomp} % provide euro and other symbols
\else % if luatex or xetex
  \usepackage{unicode-math} % this also loads fontspec
  \defaultfontfeatures{Scale=MatchLowercase}
  \defaultfontfeatures[\rmfamily]{Ligatures=TeX,Scale=1}
\fi
\usepackage{lmodern}
\ifPDFTeX\else
  % xetex/luatex font selection
\fi
% Use upquote if available, for straight quotes in verbatim environments
\IfFileExists{upquote.sty}{\usepackage{upquote}}{}
\IfFileExists{microtype.sty}{% use microtype if available
  \usepackage[]{microtype}
  \UseMicrotypeSet[protrusion]{basicmath} % disable protrusion for tt fonts
}{}
\makeatletter
\@ifundefined{KOMAClassName}{% if non-KOMA class
  \IfFileExists{parskip.sty}{%
    \usepackage{parskip}
  }{% else
    \setlength{\parindent}{0pt}
    \setlength{\parskip}{6pt plus 2pt minus 1pt}}
}{% if KOMA class
  \KOMAoptions{parskip=half}}
\makeatother
% Make \paragraph and \subparagraph free-standing
\makeatletter
\ifx\paragraph\undefined\else
  \let\oldparagraph\paragraph
  \renewcommand{\paragraph}{
    \@ifstar
      \xxxParagraphStar
      \xxxParagraphNoStar
  }
  \newcommand{\xxxParagraphStar}[1]{\oldparagraph*{#1}\mbox{}}
  \newcommand{\xxxParagraphNoStar}[1]{\oldparagraph{#1}\mbox{}}
\fi
\ifx\subparagraph\undefined\else
  \let\oldsubparagraph\subparagraph
  \renewcommand{\subparagraph}{
    \@ifstar
      \xxxSubParagraphStar
      \xxxSubParagraphNoStar
  }
  \newcommand{\xxxSubParagraphStar}[1]{\oldsubparagraph*{#1}\mbox{}}
  \newcommand{\xxxSubParagraphNoStar}[1]{\oldsubparagraph{#1}\mbox{}}
\fi
\makeatother


\usepackage{longtable,booktabs,array}
\usepackage{calc} % for calculating minipage widths
% Correct order of tables after \paragraph or \subparagraph
\usepackage{etoolbox}
\makeatletter
\patchcmd\longtable{\par}{\if@noskipsec\mbox{}\fi\par}{}{}
\makeatother
% Allow footnotes in longtable head/foot
\IfFileExists{footnotehyper.sty}{\usepackage{footnotehyper}}{\usepackage{footnote}}
\makesavenoteenv{longtable}
\usepackage{graphicx}
\makeatletter
\newsavebox\pandoc@box
\newcommand*\pandocbounded[1]{% scales image to fit in text height/width
  \sbox\pandoc@box{#1}%
  \Gscale@div\@tempa{\textheight}{\dimexpr\ht\pandoc@box+\dp\pandoc@box\relax}%
  \Gscale@div\@tempb{\linewidth}{\wd\pandoc@box}%
  \ifdim\@tempb\p@<\@tempa\p@\let\@tempa\@tempb\fi% select the smaller of both
  \ifdim\@tempa\p@<\p@\scalebox{\@tempa}{\usebox\pandoc@box}%
  \else\usebox{\pandoc@box}%
  \fi%
}
% Set default figure placement to htbp
\def\fps@figure{htbp}
\makeatother





\setlength{\emergencystretch}{3em} % prevent overfull lines

\providecommand{\tightlist}{%
  \setlength{\itemsep}{0pt}\setlength{\parskip}{0pt}}



 


\KOMAoption{captions}{tableheading}
\makeatletter
\@ifpackageloaded{caption}{}{\usepackage{caption}}
\AtBeginDocument{%
\ifdefined\contentsname
  \renewcommand*\contentsname{Table of contents}
\else
  \newcommand\contentsname{Table of contents}
\fi
\ifdefined\listfigurename
  \renewcommand*\listfigurename{List of Figures}
\else
  \newcommand\listfigurename{List of Figures}
\fi
\ifdefined\listtablename
  \renewcommand*\listtablename{List of Tables}
\else
  \newcommand\listtablename{List of Tables}
\fi
\ifdefined\figurename
  \renewcommand*\figurename{Figure}
\else
  \newcommand\figurename{Figure}
\fi
\ifdefined\tablename
  \renewcommand*\tablename{Table}
\else
  \newcommand\tablename{Table}
\fi
}
\@ifpackageloaded{float}{}{\usepackage{float}}
\floatstyle{ruled}
\@ifundefined{c@chapter}{\newfloat{codelisting}{h}{lop}}{\newfloat{codelisting}{h}{lop}[chapter]}
\floatname{codelisting}{Listing}
\newcommand*\listoflistings{\listof{codelisting}{List of Listings}}
\makeatother
\makeatletter
\makeatother
\makeatletter
\@ifpackageloaded{caption}{}{\usepackage{caption}}
\@ifpackageloaded{subcaption}{}{\usepackage{subcaption}}
\makeatother
\usepackage{bookmark}
\IfFileExists{xurl.sty}{\usepackage{xurl}}{} % add URL line breaks if available
\urlstyle{same}
\hypersetup{
  pdftitle={Analysis of class surveys},
  pdfauthor={Lucas Childs, Minu , Nathan , Anna , Bahaar },
  colorlinks=true,
  linkcolor={blue},
  filecolor={Maroon},
  citecolor={Blue},
  urlcolor={Blue},
  pdfcreator={LaTeX via pandoc}}


\title{Analysis of class surveys}
\usepackage{etoolbox}
\makeatletter
\providecommand{\subtitle}[1]{% add subtitle to \maketitle
  \apptocmd{\@title}{\par {\large #1 \par}}{}{}
}
\makeatother
\subtitle{If you want a subtitle put it here}
\author{Lucas Childs, Minu \textbf{, Nathan }, Anna \textbf{, Bahaar }}
\date{2025-10-18}
\begin{document}
\maketitle


Use this as a template. Keep the headers and remove all other text.
Overall, your report may be quite short. When it is complete, render and
then push changes to your team repository.

\subsection{Executive summary}\label{executive-summary}

Write a one-paragraph abstract summarizing what you did and your
findings. It need not be comprehensive; try to highlight the most
important or interesting outcomes.

\textbf{Lucas:} Half the class has taken 9+ upper division courses.
Correlation matrix shows that there are low correlations between upper
division course numbers and comfortability in math, stats, and
programming, the highest being 0.38 between course number and stats.
Clustering: using 5 cluster centroids, we grouped students who'd taken
6+ upper division courses into 2 groups of comfortability between 3.5
and 4 and one above 4. Then for students who'd taken 3-5 upper division
courses, we discovered 2 groups of comfortability between 3 and 3.5 and
the other above 4. Thus, on average, the clustering results show that we
can group students who've taken more upper division courses into a
category of relatively higher comfortability in math, stats, and
programming.

\subsection{Data description}\label{data-description}

Write a brief description of the dataset. Your description should cover
how data were obtained, sample characteristics, and measurements taken.
It does not need to be exhaustive, but sufficiently detailed to convey a
clear high-level understanding of the dataset. You can utilize bullet
points or tables if you feel it would help improve clarity.

Data were obtained by Google Forms survey that was administered to all
students in PSTAT 197A during the Fall of 2025.

\subsection{Questions of interest}\label{questions-of-interest}

Indicate the questions your analysis addresses. These should map
one-to-one to your findings. Don't include questions you didn't answer
or questions you started with and refined later. If you would rather
frame them as goals or tasks rather than questions that is okay; just
modify the header appropriately. However you frame what you've done, you
may only have two or three items; that is fine. Provide an itemized or
numbered list so that the reader can easily identify your objectives.

For example:

\begin{quote}
\emph{We sought to understand the relationship between coursework
preparation, self-assessed technical abilities, and comfort level with
technical skills. We also sought to understand whether the relationships
differed depending on having had prior research experience. Our analysis
addressed three questions:}

\begin{enumerate}
\def\labelenumi{\arabic{enumi}.}
\tightlist
\item
  \emph{How should the coursework histories of students in the class be
  ordered from less preparation to extensive preparation?}
\item
  \emph{Is coursework preparation associated with increased
  self-assessed technical skill and comfort within the sample?}
\item
  \emph{Does prior research experience appear to be a substitute for
  coursework preparation with respect to self-assessed technical skill
  and comfort?}
\end{enumerate}
\end{quote}

Notice that the questions are precise but not overly technical.

\subsection{Findings}\label{findings}

Summarize your results. Don't try to explain every step you took; focus
instead on providing the main data analytic outputs -- tables and
figures -- and explaining clearly what they show. Clarify any important
decisions you made in obtaining them. You can display codes if you like
but it is not necessary.

\subsubsection{Links}\label{links}

To include any hyperlinks, use \texttt{{[}display\ text{]}(url)} .




\end{document}
